%%%%%%%%%%%%%%%%%%%%%%%%%%%%%%%%%%%%%%%%%%%%%%%%%%%%%%%%%%%%%%%%%%%%%%%%%%%%%%%
% Neuroimage-like layout
\documentclass[5p]{elsarticle}
\usepackage{amsmath,amsfonts,amssymb}
\usepackage{bm}
\usepackage{algorithm}
\usepackage{algorithmic}
\usepackage{url}
\usepackage[breaklinks=true,letterpaper=true,colorlinks,bookmarks=false]{hyperref}
\usepackage[table]{xcolor}

\definecolor{deep_blue}{rgb}{0,.2,.5}
\definecolor{dark_blue}{rgb}{0,.15,.5}

\hypersetup{pdftex,  % needed for pdflatex
  breaklinks=true,  % so long urls are correctly broken across lines
  colorlinks=true,
  linkcolor=dark_blue,
  citecolor=deep_blue,
}

% Float parameters, for more full pages.
\renewcommand{\topfraction}{0.9}        % max fraction of floats at top
\renewcommand{\bottomfraction}{0.8}     % max fraction of floats at bottom
\renewcommand{\textfraction}{0.07}      % allow minimal text w. figs
%   Parameters for FLOAT pages (not text pages):
\renewcommand{\floatpagefraction}{0.6}  % require fuller float pages
%    % N.B.: floatpagefraction MUST be less than topfraction !!


\def\B#1{\mathbf{#1}}
%\def\B#1{\bm{#1}}
\def\trans{^\mathsf{T}}
% A compact fraction
\def\slantfrac#1#2{\kern.1em^{#1}\kern-.1em/\kern-.1em_{#2}}

%%%%%%%%%%%%%%%%%%%%%%%%%%%%%%%%%%%%%%%%%%%%%%%%%%%%%%%%%%%%%%%%%%%%%%%%%%%%%%%%
\begin{document}

%\title{Comparing Connectomes Between Populations}
\title{Learning and comparing functional connectomes between populations}


\author[parietal,unicog,cea]{Ga\"el Varoquaux\corref{corresponding}}
\author[child_institute]{Cameron Craddock}

\cortext[corresponding]{Corresponding author}

\address[parietal]{Parietal project-team, INRIA Saclay-\^ile de France}
\address[unicog]{INSERM, U992}
\address[cea]{CEA/Neurospin b\^at 145, 91191 Gif-Sur-Yvette}
\address[child_institute]{Child Institute, New York}

\begin{abstract}
    We are the champion... of the world

    Scope: rest and task-based studies. But focusing on fMRI.
\end{abstract}

\begin{keyword}
    Functional connectivity, connectome, group study, effective
    connectivity, fMRI, resting-state
\end{keyword}

\maketitle
%%%%%%%%%%%%%%%%%%%%%%%%%%%%%%%%%%%%%%%%%%%%%%%%%%%%%%%%%%%%%%%%%%%%%%%%%%%%%%%%

\sloppy % Fed up with messed-up line breaks
\section{Introduction}

Review paper giving technical guidelines.

%%%%%%%%%%%%%%%%%%%%%%%%%%%%%%%%%%%%%%%%%%%%%%%%%%%%%%%%%%%%%%%%%%%%%%%%%%%%%%%

\section{Estimating functional connectomes}

%------------------------------------------------------------------------------
\subsection{Defining regions}

Different strategies: 
dense/sparse: covering a large fraction of the
brain or not, hard vs soft boundaries.

\paragraph{Regions from atlases}

AAL, Harvard-oxford

\paragraph{Defining regions from the literature}
Meta-analysis

\paragraph{FMRI-based function definition}
Distinction between activation and rest.
Activation: thresholding GLM maps, or using spheres around the activation
peak.
Rest: ICA-based approaches \cite{kiviniemi2009} XXX: cite Grecius paper.
\cite{varoquaux2011}

%------------------------------------------------------------------------------
\subsection{Estimating connections}

% Disclaimer: focus on correlation/second order statistics?

\paragraph{Signal extraction}

Once again separate task from rest

Task: beta time-series, and more recent stuff, \emph{e.g.} by Mumford and
Poldrack.

Rest: importance of confounds. Filtering + and detrending.

\paragraph{Correlation and partial correlations}

Correlation, shrinkage of correlation, partial correlation, sparse iCov.

%%%%%%%%%%%%%%%%%%%%%%%%%%%%%%%%%%%%%%%%%%%%%%%%%%%%%%%%%%%%%%%%%%%%%%%%%%%%%%%

\section{Comparing connections}

%------------------------------------------------------------------------------
\subsection{Mass-univariate approaches}

%------------------------------------------------------------------------------
\subsection{Modeling between-connection dependences}

%%%%%%%%%%%%%%%%%%%%%%%%%%%%%%%%%%%%%%%%%%%%%%%%%%%%%%%%%%%%%%%%%%%%%%%%%%%%%%%

\section{Comparing Network Summary Statistics}

%%%%%%%%%%%%%%%%%%%%%%%%%%%%%%%%%%%%%%%%%%%%%%%%%%%%%%%%%%%%%%%%%%%%%%%%%%%%%%%

\section{Predictive Modeling}

%%%%%%%%%%%%%%%%%%%%%%%%%%%%%%%%%%%%%%%%%%%%%%%%%%%%%%%%%%%%%%%%%%%%%%%%%%%%%%%

\section{Functional and effective connectivity}

%------------------------------------------------------------------------------
\subsection{From correlations to structural equation modeling}

\cite{mcintosh1994}

%------------------------------------------------------------------------------
\subsection{Matching model complexity to data}

Diatribe: all model are wrong, but some are useful

Setting the cursor between bio-physical models 

\cite{mcintosh2010}

%%%%%%%%%%%%%%%%%%%%%%%%%%%%%%%%%%%%%%%%%%%%%%%%%%%%%%%%%%%%%%%%%%%%%%%%%%%%%%%

\section{Conclusion}

{
%\clearpage
\section*{References} \small \bibliographystyle{elsarticle-num-names}
\bibliography{biblio} }

%%%%%%%%%%%%%%%%%%%%%%%%%%%%%%%%%%%%%%%%%%%%%%%%%%%%%%%%%%%%%%%%%%%%%%%%%%%%%%%


\end{document}
